% ---------
%  Compile with "pdflatex hw0".
% --------
%!TEX TS-program = pdflatex
%!TEX encoding = UTF-8 Unicode

\documentclass[11pt]{article}
\usepackage{jeffe,handout,graphicx}
\usepackage{amsmath}
\usepackage[utf8]{inputenc}		% Allow some non-ASCII Unicode in source

%  Redefine suits
\usepackage{pifont}
\def\Spade{\text{\ding{171}}}
\def\Heart{\text{\textcolor{Red}{\ding{170}}}}
\def\Diamond{\text{\textcolor{Red}{\ding{169}}}}
\def\Club{\text{\ding{168}}}

\def\Cdot{\mathbin{\text{\normalfont \textbullet}}}
\def\Sym#1{\textbf{\texttt{\color{BrickRed}#1}}}

\newtheorem{claim}{Claim}

% =====================================================
%   Define common stuff for solution headers
% =====================================================
\Class{CS 498 VR}
\Semester{Fall 2017}
\Authors{2}
\AuthorOne{Cong Shen}{cshen19@illinois.edu}
\AuthorTwo{Hongting Liu}{hl25@illinis.edu}
%\Section{}

% =====================================================
\begin{document}

% ---------------------------------------------------------


\HomeworkHeader{2}{4}	% homework number, problem number

\begin{quote}
{\Large \textbf{HomeWork 2 Part 4 Written Assignment}}\\

\begin{enumerate}[1.]
    \item
    In one sentence, explain what the following homogeneous transformation accomplishes
    when applied to a point (x, y, z), in terms of yaw, pitch, roll, and translation.
    \[
        T_1 =
        \begin{bmatrix}
            \frac{1}{\sqrt{2}} & -\frac{1}{\sqrt{2}} & 0 & -1\\
            \frac{1}{\sqrt{2}} & \frac{1}{\sqrt{2}} & 0 & 2\\
            0 & 0 & 1 & 0\\
            0 & 0 & 0 & 1
        \end{bmatrix}
        \begin{bmatrix}
            x\\
            y\\
            z\\
            1
        \end{bmatrix}
    \]
    \item
    Write out the $4 \times 4$ homogeneous transformation $T_2$
    , when applied to a point $(x, y ,z)$ in
    global coordinate frame, translates the point by $(3, 0, 2) ^ {T}$ , then followed by a pitch of $45$
    degrees. Your answer need not be simplified, and may be represented as a single
    matrix or the product of two or more matrices.
    \item
    We would like to reverse the transformation applied by $T_1T_2$ , that is, write out
    $(T_2 T_1)^ T$ . Your answer need not be simplified, and may be represented as a single
    matrix or the product of two or more matrices.
    \item
    Write out the Quaternion equivalent to the rotations in $T_1$ and $T_2$ as $q_1$ and $q_2$. 
    Then calculate the product, that is, $q_1 \circ q_2$ (Hint: Steve’s book may be a good source of
    reference)
\end{enumerate}
\end{quote}
\hrule

\begin{solution}

\begin{enumerate}[1.]
    \item 
        This transformation is rotatation by $R$ and then translation by $T$, where\\
        \[
            R =
            \begin{bmatrix}
                \frac{1}{\sqrt{2}} & -\frac{1}{\sqrt{2}} & 0 & 0\\
                \frac{1}{\sqrt{2}} & \frac{1}{\sqrt{2}} & 0 & 0\\
                0 & 0 & 1 & 0\\
                0 & 0 & 0 & 1
            \end{bmatrix}
            \quad
            T = 
            \begin{bmatrix}
                1 & 0 & 0 & -1\\
                0 & 1 & 0 & 2\\
                0 & 0 & 1 & 0\\
                0 & 0 & 0 & 1
            \end{bmatrix}
        \]
        $R$ implies that the rotation is a roll by $\frac{\pi}{4}$
    \item
        \[
            T_2 =
            \begin{bmatrix}
                1 & 0 & 0 & 3\\
                0 & 1 & 0 & 0\\
                0 & 0 & 1 & 2\\
                0 & 0 & 0 & 1
            \end{bmatrix}
            \begin{bmatrix}
                1 & 0 & 0 & 0\\
                0 & \frac{1}{\sqrt{2}} & -\frac{1}{\sqrt{2}} & 0\\
                0 & \frac{1}{\sqrt{2}} & \frac{1}{\sqrt{2}} & 0\\
                0 & 0 & 0 & 1
            \end{bmatrix}
            =
            \begin{bmatrix}
                1 & 0 & 0 & 3\\
                0 & \frac{1}{\sqrt{2}} & -\frac{1}{\sqrt{2}} & -\sqrt{2}\\
                0 & \frac{1}{\sqrt{2}} & \frac{1}{\sqrt{2}} & -\sqrt{2}\\
                0 & 0 & 0 & 1
            \end{bmatrix}
        \]
    \item
    \[
        {T_1}^{-1} =
        \begin{bmatrix}
            \frac{1}{\sqrt{2}} & \frac{1}{\sqrt{2}} & 0 & -\frac{1}{\sqrt{2}} \\
            -\frac{1}{\sqrt{2}} & \frac{1}{\sqrt{2}} & 0 & -\frac{3}{\sqrt{2}} \\
            0 & 0 & 1 & 0\\
            0 & 0 & 0 & 1
        \end{bmatrix}
        \quad
        {T_2}^{-1} =
        \begin{bmatrix}
            1 & 0 & 0 & -3\\
            0 & \frac{1}{\sqrt{2}} & \frac{1}{\sqrt{2}} & 0 \\
            0 & -\frac{1}{\sqrt{2}} & \frac{1}{\sqrt{2}} & -2 \\
            0 & 0 & 0 & 1
        \end{bmatrix}
    \]
    \[
        {(T_2T_1)}^{-1} = {T_1}^{-1} {T_2}^{-2} =
        \begin{bmatrix}
            \frac{1}{\sqrt{2}} & \frac{1}{2} & \frac{1}{2} & -2\sqrt{2}\\
            -\frac{1}{\sqrt{2}} & \frac{1}{2} & \frac{1}{2} & 0\\
            0 & -\frac{1}{\sqrt{2}} & \frac{1}{\sqrt{2}} & -2 \\
            0 & 0 & 0 & 1
        \end{bmatrix}
    \]
    \item
    \[
        q1 = 
        \begin{bmatrix}
            cos\frac{\pi}{8} & 0 & 0 & sin\frac{\pi}{8}
        \end{bmatrix}
        \quad
        q2 =
        \begin{bmatrix}
            cos\frac{\pi}{8} & sin\frac{\pi}{8} & 0 & 0
        \end{bmatrix} 
    \]
    \[
        q1 \circ q2 = 
        \begin{bmatrix}
            0.85355 & 0.35355 & 0.14645 & 0.35355
        \end{bmatrix}
    \]
\end{enumerate}

\end{solution}


\end{document}
